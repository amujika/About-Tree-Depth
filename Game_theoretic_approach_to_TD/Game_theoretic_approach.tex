\section{Game Theoretic approach to Tree-Depth}

\subsection{Defining the game}
\paragraph{}
The k-step selection-deletion game is played by Alice and Bob on a graph G. The game is played by turns as follows:
\begin{itemize}
  \item First, Alice selects a connected component of the graph, and the rest of the components are deleted.
  \item Then, Bob deletes a node from the remaining graph.
\end{itemize}

If Bob deletes the last node at the k-th round or earlier, he is said to win the k-step selection-deletion game. Otherwise, Alice wins.

\subsection{Bob's winning strategy}
\begin{lemma}
Let G be a graph and let F be a rooted forest of height at most t such that G $\subseteq$ clos(F). Then Bob has a winning strategy for the (t+1)-step selection-deletion game.
\end{lemma}
\begin{proof}
Because of lemma~\ref{lema:min-ET} we know an elimination forest Y exists such that height(Y) $\leq$ height(F).
We will proof this by induction on the height of Y.
\begin{itemize}
  \item \textbf{Base case}: If height(Y) = 0, then every component of G will have a single vertex, so it's clear that Bob will win the 1-step selection-deletion game.
  \item \textbf{Induction}: Let $G_i \subseteq G$ be the component Alice chooses, then $Y_i$ exists such that $Y_i$ is an elimination tree belonging to Y, $G_i \subseteq clos(Y_i)$ and obviously $height(Y_i) \leq height(Y) \leq t$. Bob will delete v, the root of $Y_i$. This will leave us with $G' = G_i - v$ as the new graph. If we consider the sons of v the new roots in $Y' = Y_i - v$, then $G' \subseteq clos(Y')$ because of how the elimination trees are built. As $height(Y') \leq t-1$, we can assume by induction that Bob has a winning strategy in t rounds for $G'$, which together with the strategy for the first round we have just defined makes a winning strategy for Bob in the (t+1)-step selection-deletion game on the graph G.
\end{itemize}
$\qed$
\end{proof}

\subsection{Alice's winning strategy}