\section{Cycle rank}

\subsection{Defining cycle rank}

\paragraph{}
Cycle rank is a numerical invariant in a directed graph which is closely related to the tree depth of an undirected graph.
\begin{definition}
The cycle rank of a digraph G, denoted by r(G) is defined as follows:

\begin{itemize}
\item If G is acyclic, then r(G) = 0.
\item If G is strongly connected and E $\neq \emptyset$, then r(G) = 1 + $min_{v \in V(G)} \left\{r(G-v)\right\}$.
\item If G is not strongly connected and contains at least a cycle, then r(G) is the maximum cycle rank among all strongly connected components of G.
\end{itemize}

The E $\neq \emptyset$ condition is necessary because the graph with a single vertex and no edges is both acyclic and strongly connected.

\end{definition}

\subsection{Directed elimination tree}

\paragraph{}
Similar to the notion of elimination trees in undirected graphs, we have directed elimination trees on digraphs.

\begin{definition}

A directed elimination tree for a nontrivially strongly connected digraph G = (V, E) is a rooted tree $T = (\nabla, \xi)$. T can be defined recursively as follows:

\begin{itemize}
\item (v, V) is the root of T for some v $\in$ V.
\item For every nontrivially strongly connected component in G - v, $Y_1, \ldots, Y_j$ a directed elimination tree for G[$Y_i$] is created for every $1 \leq i \leq j$. The roots of these trees are conneced to the root of T.
\end{itemize}
\label{definition:DET}
\end{definition}

We can extend the concept of directed elimination trees to directed elimination forests. For a digraph G with k $\geq$ 0 nontrivially strongly connected components $C_1, \ldots, C_k$, a rooted forest is the union of the rooted trees of G[$C_i$] for 1 $\leq$ i $\leq$ k.

\begin{observation}
In the second step of the definition~\ref{definition:DET}, what we actually do is creating a directed elimination forest for G-v and connect its roots to (v, V).
\label{observation:eforest}
\end{observation}

\begin{lemma}
Let F be directed elimination forests of minimum height for a digraph G = (V, E). Then, r(G) = height(F). 
\end{lemma}
\begin{proof}
We will proof this by induction on the number of vertices of G.
\begin{itemize}
  \item \textbf{Base case}: If G is acyclic, then r(G) = 0, height(F) is obviously 0 because F is the null graph. We consider the null graph acyclic.
  \item \textbf{Induction}: If G is nontrivially strongly connected, then v $\in$ V exists, such that r(G) = 1 + r(G-v). Let (v, V) be the root of F, then height(F) = 1 + height(F') where F' is a directed elimination forest of G-v because of observation~\ref{observation:eforest}. By induction we can assume that r(G-v) equals the minimun height of a directed elimination forest F'
\end{itemize}
\end{proof}