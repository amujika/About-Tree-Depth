\section{Cycle rank}

\subsection{Defining cycle rank}

\paragraph{}
Cycle rank is a numerical invariant in a directed graph which is closely related to the tree depth of an undirected graph.
\begin{definition}
The cycle rank of a digraph G, denoted by r(G) is defined as follows:

\begin{itemize} 
\item If G is acyclic, then r(G) = 0.
\item If G is nontrivial strongly connected, then r(G) = 1 + $min_{v \in V(G)} \left\{r(G-v)\right\}$.
\item If G is not strongly connected and contains at least a cycle, then r(G) is the maximum cycle rank among all nontrivial strongly connected components of G.
\end{itemize}

The graph with a single node and no edges is considered trivially strongly connected, while the graph with a single node and a loop is considered nontrivially strongly connected.

\end{definition}

\subsection{Directed elimination forest}

\paragraph{}
Similar to the notion of elimination forests in undirected graphs, we have directed elimination forests on digraphs.

\begin{definition}

A directed elimination forest for a digraph G is a rooted forest F. F can be defined recursively as follows:

\begin{itemize}
\item For the k $\geq$ 0 non trivially strongly connected components of G, $Y_1, \ldots, Y_k$, ($v_i$, $Y_i$) are the roots in F, where $v_i \in Y_i$ and 1 $\leq$ i $\leq$ k.
\item For each ($v_i$, $Y_i$), a directed elimination forest is created for G[$Y_i$] - $v_i$ and the roots of that forest are the children of ($v_i$, $Y_i$) in F.
\end{itemize}
\label{definition:DET}
\end{definition}

\begin{lemma}
Let F be directed elimination forests of minimum height for a digraph G = (V, E). Then, r(G) = height(F). 
\end{lemma}
\begin{proof}
We will proof this by induction on the number of vertices of G.
\begin{itemize}
  \item \textbf{Base case}: If G is acyclic, then r(G) = 0, height(F) is 0 because we assume that the height of the empty tree is 0.
  \item \textbf{Induction}: If G is nontrivially strongly connected, then v $\in$ V exists, such that r(G) = 1 + r(G-v). Let (v, V) be the root of F, then height(F) = 1 + height(F') where F' is any directed elimination forest of G-v because of definition~\ref{definition:DET}. If we consider F' to be the directed elimination forest of G-v of minimum height, by induction we can assume that r(G-v) = 1 + height(F'). So, r(G) = 1 + r(G-v) = 1 + height(F') = height(F).
  
  If G is not strongly connected but it has at least a cycle, then, for every X that is a strongly connected component of G by induction we can assume that r(G[X]) = height($F_X$) where $F_X$ is the directed elimination tree of minimum height for G[X]. Because r(G) is the maximum among all r(G[X]) and the height(F) is the maximum among all height($F_X$), r(G) = height(F).
\end{itemize}
\end{proof}