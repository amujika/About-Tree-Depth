\section{Introduction to Graph Theory}

\subsection{Definition of a graph}
\paragraph{}
A graph is defined as a pair of sets G = (V, E), such that E $\subseteq$ $\left\{\{a, b\} \mid a,b \in V\right\}$. The members of V are called vertices and the ones of E edges. Take into account, that the vertices can be anything, they can even be sets themselves. The usual way to draw a graph is by representing the vertices as individual points and for each edge, draw a link between both elements of that edge. The shape in which a graph is drawn is irrelevant, it will contain the same information.

\begin{figure}[h]
\begin{tikzpicture}[node distance   = 4 cm]]

  \tikzset{main node/.style={
   shape = circle,
   fill  = blue!15,
   draw,
   font  = \sffamily\small\bfseries}
   }
  \tikzset{EdgeStyle/.style= {thick}}

  \node[main node] (1) [label=A] {};
  \node[main node] (2) [below left of=1] [label=B] {};
  \node[main node] (3) [below right of=2] [label=C] {};
  \node[main node] (4) [below right of=1] [label=D] {};
  \node[main node] (5) [right of=4] [label=E] {};
  \node[main node] (6) [right of=3] [label=F] {};

  \draw[EdgeStyle](1) to (2);
  \draw[EdgeStyle](1) to (4);
  \draw[EdgeStyle](2) to (3);
  \draw[EdgeStyle](2) to (4);
  \draw[EdgeStyle](4) to (5);

\end{tikzpicture}
\caption{A graph with V = \{A, B, C, D, E, F\} and E = \{\{A, B\}, \{A, D\}, \{B, D\}, \{B, C\}, \{D, E\}\}}
\end{figure}

\subsection{Definitions for undirected graphs}
For a graph G = (V, E).
\begin{description}
 \item[Adjacency] \hfill \\
  a, b $\in$ V are said to be adjacent in G if \{a, b\} $\in$ E.
 \item[Path] \hfill \\
  A path $a_1, a_2, \ldots, a_n$ is a series of vertices in V such that if $2 \leq i \leq n$, then $a_{i-1}$ is adjacent to $a_i$. n is the length of such a path.
 \item[Cycle] \hfill \\
  A cycle is a path of the form a, $\ldots$, a of length greater than 1.
 \item[Subgraph] \hfill \\
 G' is a subgraph of G, expressed as G' $\subseteq$ G, if V(G') $\subseteq$ V(G) and E(G') $\subseteq$ E(G). G' $\subset$ G means that G' $\subseteq$ G but V(G') $\neq$ V(G) or E(G') $\neq$ E(G).
 \item[Connected component] \hfill \\
  A connected component G' of G is a subgraph of G such that a path exists between any two vertices of G' and no H exists such that G' $\subset$ H and H is a connected component.
  \item[Tree] \hfill \\ 
  A tree is a graph with a single connected component and no cycle.
  \item[Forest] \hfill \\
  A forest is a graph such that every connected component is a tree.
  \item[Rooted Tree] \hfill \\ 
  A rooted tree is a tree with a special node that is called the root.
  \item[Rooted Forest] \hfill \\ 
  A rooted forest is a graph such that every connected component is a rooted tree.
  \item[Height of a node] \hfill \\ 
  The height of a node in a rooted tree is the lenght of the path from that node to the root. The height of the root itself is 1. In a rooted forest, the height of a node is its height in the rooted tree it belongs to.
  \item[Height of rooted forest] \hfill \\ 
  The height of a rooted forest is the maximum height of any of its nodes.
\end{description}

