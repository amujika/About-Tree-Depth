\section{Introduction to Graph Theory}

\subsection{Definition of a graph}
\paragraph{}
A graph is defined as a pair of sets G = (V, E), such that E $\subseteq$ $V^2$. The members of V are called vertices and the ones of E edges. Take into account, that the vertices can be anything, they can even be sets themselves. The usual way to draw a graph is by representing the vertices as individual points and for each edge, draw a link between both elements of that edge. The shape in which a graph is drawn is irrelevant, it will contain the same information.

\begin{figure}[h]
\begin{tikzpicture}[node distance   = 4 cm]]

  \tikzset{main node/.style={
   shape = circle,
   fill  = blue!15,
   draw,
   font  = \sffamily\small\bfseries}
   }
  \tikzset{EdgeStyle/.style= {thick}}

  \node[main node] (1) [label=A] {};
  \node[main node] (2) [below left of=1] [label=B] {};
  \node[main node] (3) [below right of=2] [label=C] {};
  \node[main node] (4) [below right of=1] [label=D] {};
  \node[main node] (5) [right of=4] [label=E] {};
  \node[main node] (6) [right of=3] [label=F] {};

  \draw[EdgeStyle](1) to (2);
  \draw[EdgeStyle](1) to (4);
  \draw[EdgeStyle](2) to (3);
  \draw[EdgeStyle](2) to (4);
  \draw[EdgeStyle](4) to (5);

\end{tikzpicture}
\caption{A graph with V = \{A, B, C, D, E, F\} and E = \{\{A, B\}, \{A, D\}, \{B, D\}, \{B, C\}, \{D, E\}\}}
\end{figure}

\subsection{Connectivity}
\paragraph{}
An essential concept in graph theory is adjacency. Two vertices x, y $\in$ V are said to be adjacent in G if and only if \{x, y\} $\in$ E.

