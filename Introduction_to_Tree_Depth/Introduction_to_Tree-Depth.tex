\section{Introduction to Tree Depth}

\subsection{Basic definitions}

\paragraph{}
Vertex x is said to be the ancestor of y in a rooted forest F, if and only if x belongs to the path between y and the root of the component to which x belongs, y included.
\paragraph{}
The closure of a rooted forest F, expressed as C = clos(F), is defined as follows:
\begin{itemize}
  \item V(C) = V(F)
  \item E(C) = \{\{x, y\} : x is an ancestor of y in F, x $\neq$ y \}
\end{itemize}
\begin{figure}[h]
\begin{tikzpicture}[node distance   = 2 cm]]

  \tikzset{F node/.style={
   shape = circle,
   fill  = blue!15,
   draw,
   font  = \sffamily\small\bfseries}
   }
  \tikzset{C node/.style={
   shape = circle,
   fill  = red!15,
   draw,
   font  = \sffamily\small\bfseries}
   }
  \tikzset{EdgeStyle/.style= {thick}}
%F graph
  \node[F node] (F1) [label=Root] {};
  \node[F node] (F2) [below of=F1] {};
  \node[F node] (F3) [below right of=F2] {};
  \node[F node] (F4) [below left of=F2] {};
  \node[F node] (F5) [below  of=F4] {};
  \node[F node] (F6) [right of = F1][label=Root] {};
 
  \draw[EdgeStyle](F1) to (F2);
  \draw[EdgeStyle](F2) to (F3);
  \draw[EdgeStyle](F2) to (F4);
  \draw[EdgeStyle](F4) to (F5);
  
%C graph  
  \node[C node] (C1) [right = 8cm of F1] {};
  \node[C node] (C2) [below of=C1] {};
  \node[C node] (C3) [below right of=C2] {};
  \node[C node] (C4) [below left of=C2] {};
  \node[C node] (C5) [below  of=C4] {};
  \node[C node] (C6) [right of=C1] {};
   
  \draw[EdgeStyle](C1) to (C2);
  \draw[EdgeStyle](C1) to (C3);
  \draw[EdgeStyle](C1) to (C4);
  \draw[EdgeStyle][bend right=45](C1) to (C5);
  \draw[EdgeStyle](C2) to (C3);
  \draw[EdgeStyle](C2) to (C4);
  \draw[EdgeStyle](C2) to (C5);
  \draw[EdgeStyle](C4) to (C5);

\end{tikzpicture}

\caption{The blue graph at the left is a rooted forest F, the red graph at the right represents clos(F).}
\end{figure}

\subsection{Tree Depth}

\paragraph{Definition:}The tree-depth td(G) of a graph G is the minimum height of a rooted forest F such that G $\subseteq$ clos(F)

\begin{figure}[H]
\begin{tikzpicture}[node distance   = 2 cm]]

  \tikzset{G node/.style={
   shape = circle,
   fill  = blue!15,
   draw,
   font  = \sffamily\small\bfseries}
   }
  \tikzset{T node/.style={
   shape = circle,
   minimum size = 0.8cm,
   fill  = red!15,
   draw,
   font  = \sffamily\small\bfseries}
   }
  \tikzset{EdgeStyle/.style= {thick}}
  \tikzset{F EdgeStyle/.style= {
  thick,
  dashed}}
%G graph
  \node[G node] (G1) {A};
  \node[G node] (G2) [right = 4cm of G1] {B};
  \node[G node] (G3) [below = 4cm of G2] {C};
  \node[G node] (G4) [left = 4cm of G3] {D};
  \node[G node] (g1) [below right of=G1] {a};
  \node[G node] (g2) [right of=g1] {b};
  \node[G node] (g3) [below of=g2] {c};
  \node[G node] (g4) [left of=g3] {d};
 
  \draw[EdgeStyle](G1) to (G2);
  \draw[EdgeStyle](G2) to (G3);
  \draw[EdgeStyle](G3) to (G4);
  \draw[EdgeStyle](G4) to (G1);
  
  \draw[EdgeStyle](g1) to (g2);
  \draw[EdgeStyle](g2) to (g3);
  \draw[EdgeStyle](g3) to (g4);
  \draw[EdgeStyle](g4) to (g1);
  
  \draw[EdgeStyle](G1) to (g1);
  \draw[EdgeStyle](G2) to (g2);
  \draw[EdgeStyle](G3) to (g3);
  \draw[EdgeStyle](G4) to (g4);
  
%F graph
  \node[T node] (F1) [right = 4cm of G2] {B};
  \node[T node] (F2) [below of=F1] {a};
  \node[T node] (F3) [below of=F2] {D};
  \node[T node] (F4) [below left of=F3] {A};
  \node[T node] (F5) [below right of=F3] {c};
  \node[T node] (F6) [below of=F5] {d};
  \node[T node] (F7) [below right of=F5] {b};
  \node[T node] (F8) [right of=F5] {C};
  
  \draw[EdgeStyle](F1) to (F2);
  \draw[EdgeStyle](F2) to (F3);
  \draw[EdgeStyle](F3) to (F4);
  \draw[EdgeStyle](F3) to (F5);
  \draw[EdgeStyle](F5) to (F6);
  \draw[EdgeStyle](F5) to (F7);
  \draw[EdgeStyle](F5) to (F8);
  
%clos(F) edgesF EdgeStyle
  \draw[F EdgeStyle][bend right=45](F1) to (F3);
  \draw[F EdgeStyle][bend right=30](F1) to (F4);
  \draw[F EdgeStyle][bend left=45](F1) to (F5);
  \draw[F EdgeStyle][bend right=45](F1) to (F6);
  \draw[F EdgeStyle][bend left=20](F1) to (F7);
  \draw[F EdgeStyle][bend left=45](F1) to (F8);
  \draw[F EdgeStyle][bend right=30](F2) to (F4);
  \draw[F EdgeStyle][bend left=10](F2) to (F7);
  \draw[F EdgeStyle][bend left=30](F2) to (F8);
  \draw[F EdgeStyle][bend left=30](F2) to (F5);
  \draw[F EdgeStyle][bend right=45](F2) to (F6);
  \draw[F EdgeStyle][bend right=30](F3) to (F6);
  \draw[F EdgeStyle][bend right=50](F3) to (F7);
  \draw[F EdgeStyle][bend left=20](F3) to (F8);
\end{tikzpicture}

\caption{The graph G and tree T are in the left and right respectively. The doted edges in T, represent the clos(T). Because G $\subseteq$ clos(T), we know that td(G) is at most 4.\label{fig:3d-cube}}
\end{figure}
\paragraph{}
The tree depth of a graph G is a numerical invariant of a graph. In other words, the tree depth is a property that depends only  on the abstract structure of a graph, not in its	 representation.	

\subsection{Elimination tree}
\paragraph{}
An elimination tree Y of a connected graph G is defined recursively as follows:
\begin{itemize}
  \item If V(G) = \{x\} then Y is just \{x\}.
  \item Otherwise, r $\in$ V(G) is chosen as the root of Y and an elimination tree is created for each component of G-r. r will be connected to the root of each of these trees.
\end{itemize}
The tree T in Figure~\ref{fig:3d-cube} is an elimination tree for the graph G.
\paragraph{} 