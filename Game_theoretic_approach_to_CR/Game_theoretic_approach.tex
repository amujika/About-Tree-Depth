\section{Game Characterization of Cycle Rank}

\subsection{Definitions}
\paragraph{}
For this section, we will assume all our graphs are directed and contain no self loops. We will also need some basic definitions before we can start talking about the games we will use to define cycle rank.
\begin{description}
 \item[Initial component] \hfill \\
H $\subseteq$ G is an initial component of G if H is a strongly connected component and there are no edges from G\textbackslash H to H.
 \item[Successor-closed] \hfill \\
H $\subseteq$ G is an successor-closed of G if H is there are no edges from H to G\textbackslash H.
\item[Power set] \hfill \\
The power set of a set S, denoted by $\mathcal{P}$(S), is the set of all possible subsets of S.
\item[String] \hfill \\
A string is a sequencce of elements $a_1, \ldots, a_n$ such that all $a_i$ belong to the same set. That set is called the alphabet.
\item[Length] \hfill \\
The lenght of a string  A = $a_1, \ldots, a_n$, denoteb by $|$A$|$ is n. The lenght of the empty string is 0.
\item[Concatenation] \hfill \\
The concatenation of two string A =  $a_1, \ldots, a_n$ and B =  $b_1, \ldots, b_k$, denoted by A$\cdot$B is $a_1, \ldots, a_n,b_1, \ldots, b_k$
\item[V*] \hfill \\
V* is the set of all possible finite words over the set V, including the empty word.
\item[V$^{<k}$] \hfill \\
V$^{<k}$ is the set of all possible finite words over the set V of lenght smaller than k.
\item[Prefix] \hfill \\
X $\in$ V* is a prefix of Y $\in$ V*, denoted by X$\preceq$Y  if Z $\in$ V* exists such that Y = X$\cdot$Z.
\item[String to set] \hfill \\
For a string S = $a_1, \ldots, a_n$, $\{|S|\}$ denotes the set $\{a_1, \ldots, a_n\}$.
\item[Symmetric difference] \hfill \\
For two sets A and B their symetric difference, expressed like A$\Delta$B is (A$\cup$B)\textbackslash (A $\cap$B).
\end{description}

\subsection{Game description}
\paragraph{}
We will use a cops and robbers game played on a graph G, where the cops will try to catch a robber. In each step of the game the cops can either place a searcher on a node or remove only the most recently placed searcher. This is why it's called a LIFO search. The different variants of the game depend on whether on not the robber is invisible and if he can move only in strongly connected components or in any directed paths.

For a digraph G, the state of the game is described by a pair (X, R). X $\in$ V* is the position of the cops and the order in which they were added. The meaning of R depends on the game variant we are talking about, but it will always be an induced subgraph of G\textbackslash \{$|$X$|$\}. We will introduce the 4 game variants now:
\begin{description}
\item[Invisible] \hfill \\
The position of the robber is unknown, so R represents all possible places where the robber may be. R is a succesor closed in G\textbackslash \{$|$X$|$\}.
\item[Invisible strongly connected component] \hfill \\
The position of the robber is unknown, so R represents all possible places where the robber may be. R
\end{description}