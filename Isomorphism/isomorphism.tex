\section{Isomorphism}
\begin{definition}
Let $G$ be a connected graph, $T$ an elimination tree of minimal height of $G$ and $P = p_1,\ldots, p_n$ a nonempty path in $T$ such that $p_1$ is the root of $T$. For a triple of the form ($G$, $T$, $P$):
  
\begin{itemize}
 \item $T_P = \{T_{p1}, \ldots, T_{pk}\}$ is the set of connected trees formed by descendants of $p_n$ in $T$.
 \item $P_1, \ldots, P_k$ will be the paths from the root of T to each of the roots of $T_P$.
 \item $G_P$ will be the graph induced by the descendants of $p_n$ and $p_n$ itself.
 \item For $u,v \in V(G), E_G(u, v)$ returns 1 if $\{u, v\} \in E(G)$ and 0 otherwise.
 \end{itemize}
We will now proceed to define an ordering on such triples.
\end{definition}
 
\begin{definition}
Let ($G, T, P$), ($H, Y, S$) be two triples of the form we have just defined such that $|V(G)| = |V(H)|$, $height(T) = height(Y)$ and $|P| = |S|$. We will say ($G, T, P$) $<$ ($H, Y, S$) if any of the following holds:
 
\begin{itemize}
\item $|G_P| < |H_S|$.
\item $|G_P| = |H_S|$ and $|T_P| < |Y_S|$.
\item $|G_P| = |H_S|$, $|T_P| = |Y_S|$ and $(E_G(p_1, p_n), \ldots, E_G(p_{n-1}, p_n))$ \\ $ < $ $(E_H(s_1, s_n), \ldots, E_H(s_{n-1}, s_n))$ lexicographically.
\item $|G_P| = |H_S|$, $|T_P| = |Y_S|$, $E_G(p_i, p_n) = E_H(s_i, s_n)$ $\forall i = 1, \ldots, n-1$ and $((G, T, P_1), \ldots, (G, T, P_k)) < ((H, Y, S_1), \ldots, (H, Y, S_k))$, where each list is ordered by this relation.
\end{itemize}
 
\begin{lemma}
For two triples ($G, T, P$), ($H, Y, S$), if neither ($G, T, P$) $<$ ($H, Y, S$) nor ($H, Y, S$) $<$ ($G, T, P$), then a  bijection $\phi$ exists such that $\forall v \in P \cup V(G_P)$ and $\forall u \in V(G_P)$, \{$u$, $v$\} $\in$ E($G$) $\iff$ \{$\phi(u)$, $\phi(v)$\} $\in$ E($H$) and $\phi(p_i) = s_i$.
\end{lemma}
 
\begin{proof}
We will proof this by reverse induction on the size of the paths.

Base case: 
If $|V(G_P)| = |V(H_S)| = 1$, then there is only one possible $\phi$. This $\phi$ obviously preserves the conditions mentioned in the lemma because of the third condition of the $<$ operator.

Induction:
By induction a $\phi_i$ exists from each ($G, T, P_i$) to each ($H, Y, S_i$). We can build a $\phi$ from ($G, T, P$) to ($H, Y, S$) that preserves the conditions mentioned in the Lemma simply by joining the different $\phi_i$.
\end{proof}
 
\end{definition}